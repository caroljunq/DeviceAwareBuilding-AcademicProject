\chapter{Conclusão}
\label{chap:Conclusao}


Esse projeto teve como objetivo o aprofundamento na área de Internet das Coisas,
especialmente nas características de desenvolvimento local e independente que
foram encontradas durante a construção da aplicação localizadora de contexto de
dispositivos. Esta aplicação foi construída, instalada e testada no prédio
piloto fornecendo um ambiente real onde pôde-se avaliar as reais capacidades de
um sistema localizador desse gênero.

Confirmou-se que um sistema de
geolocalização que utiliza somente FSPL com RSS tem poucas chances de ser
preciso. Porém, com as mesmas ferramentas, pôde-se construir uma rede de sensores
onde cada nó foi responsável por monitorar um contexto (sala, área ou parte de um
prédio) e, neste sentido, os dispositivos puderam ser associados ao contexto e a
localização do sensor, efetivamente identificando a localização deles
e seus portadores dentro de um modelo lógico do prédio.

Para que a implementação possa ser replicada, o custo associado foi determinado
como sendo de R\$ 324,85 por sensor onde é necessário um sensor por contexto além
de uma estrutura de rede já existente. Analogamente, o custo total do projeto
piloto foi definido como R\$ 995\footnote{Valor total das aquisições no
MercadoLivre.com} para os custos de \emph{harware} incluindo os protótipos e 400
horas\footnote{Estimado de 200 funcionalidades (\emph{git commit}) implementas
com média de 2 horas de implementação cada} de desenvolvimento.

Quanto ao estado da arte do ramo de Internet das Coisas foram identificadas
algumas características que necessitam destaque:

\begin{alineas}
	\item Num ambiente onde
	idealmente todos os objetos e coisas estão conectados, poucos padrões globais
	existem para garantir essa conexão, tanto a nível jurídico (direitos e
	obrigações de fabricantes, desenvolvedores e usuários) quanto técnico
	(protocolos de comunicação, arquiteturas e recomendações unificados);

	\item Muitos produtos e soluções são muito jovens e carecem amadurecimento,
	especialmente no mercado residencial e comercial, onde as exigências de
	segurança, padronização, conformidade legal e disponibilidade são inferiores
	às encontradas no ramo industrial;

	\item Foi encontrada uma dificuldade durante a aquisição das plataformas, pois
	sem uma orientação de um profissional da área de sistemas embarcados, a
	comparação e escolha de uma plataforma é uma tarefa muito complexa;

	\item Comparada com a comunidade de desenvolvedores \emph{Web}, a comunidade
	de desenvolvedores IoT é muito jovem e não desenvolveu ferramentas para
	construir, compartilhar e reutiliazar projetos de maneira eficiente.
\end{alineas}


\section{Resultados para comunidade e trabalhos futuros}
\label{sec:trab-futuros}

Durante a exploração do tema, foram encontradas diversas implementações de
localizadores baseados em \emph{Wi-Fi}, mas a implementação aqui executada mais se
assemelha com a de \citeonline{Ferreira2016} onde a mesma plataforma
(Raspberry Pi, porém na sua versão 2 modelo B+), tipo de adaptador (\emph{Wi-Fi} USB)
e \emph{software} (\emph{TShark}) foram utilizadas. A principal diferença são os
objetivos, enquanto a localização que \citeonline{Ferreira2016} buscou é do tipo
geográfica, neste trabalho buscou-se o objetivo mais simples de encontrar o grau
de presença do dispositivo no mesmo contexto (sala) do sensor. Outras diferenças
são que alguns desafios propostos por \citeonline{Ferreira2016} foram atacados
com certo nível de sucesso, entre eles: mais de um dispositivo sensor,
coleta e processamento simultâneos (\emph{online}) e registro do histórico.
Outros que não renderam frutos também merecem atenção, como a exploração
adicional da plataforma ESP8266 que aqui propõem-se como trabalho futuro para a
construção do mesmo sensor a um menor custo.
