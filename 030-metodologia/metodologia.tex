
\chapter{Método de Pesquisa}
\label{chap:Método de Pesquisa}

Abordagens para medir distâncias através de redes sem fio \emph{Wi-Fi}
\cite{bahillo2009ieee} e \emph{Bluetooth} já existem e, propor novas maneiras
não é o foco deste trabalho. Utilizando essas técnicas, constitui-se uma
rede de nós sensores colaborativos fixos no ambiente onde deseja-se obter a
localização dos dispositivos. As informações de distância são compartilhadas
entre os nós para maior precisão da informação.

Para a implementação, utilizou-se os \emph{softwares} de maior
destaque recentemente nos ramos de comunicação de baixa energia (\emph{MQTT}),
serviços \emph{Web} para geolocalização (\emph{Google Maps}) e publicação
(\emph{NodeJS}), além de \emph{softwares} para medição da distância sem
interferir na comuncação (\emph{Sniffing}) e das plataformas de
\emph{hardware} disponíveis e recomendadas para IoT com capacidade
\emph{Wi-Fi} (\emph{Raspberry Pi 3} e \emph{ESP-8266}).

Mesmo com a grande quantidade de dispositivos já conectados são poucos os
documentos descrevendo boas práticas para concepção, construção e manutenção de
aplicações IoT, especialmente sobre os cuidados tomados quanto a segurança e
análise de custos para a implementação e manutenção. Além
disso, a falta de referências neste sentido é agravada quando considera-se a
implementação no interior do estado de São Paulo. Nesta região, poucas são as
organizações atualizadas neste tema, levando a uma falta enorme de conteúdo
escrito na linguagem local além de serviços e produtos disponíveis para
construção de uma plataforma completa e competitiva na região.

Devido a falta de conteúdo e instrução, utiliza-se prototipagem ágil neste
projeto, uma vez que esta metodologia de desenvolvimento é recomendada para
projetos cujas especificações e definições não são claras, demandando muitas
modificações das mesmas durante a etapa de execução. Esse método entra em
contraste com metodologias clássicas, como a cascata, que apesar de previsíveis,
não reagem bem a ambientes de extrema incerteza.

Mais especificamente, utiliza-se uma variante da metodologia \emph{Scrum}
\cite{James2016} que foi adaptada para o projeto. Nela, foram executadas
iterações de uma semana em que a cada iteração, uma nova versão melhorada do
produto completo (\emph{hardware}, \emph{software}, documentação e
resultados) foi feita.

Dentro de cada iteração, as camadas da aplicação IoT foram escolhidas,
implementadas, justificadas e avaliadas, sendo parte do processo registrado sob forma de
vídeo (Youtube).

A cada iteração, cumpriu-se parte ou todo de cada objetivo proposto no trabalho,
levando o projeto gradualmente para um estágio de completude.
Cada iteração teve como foco os objetivos a seguir, sendo seus resultados
utilizados para tomar e justificar decisões durante a execução do
projeto bem como servir de posterior documentação. Os objetivos de cada iteração
são:

\begin{alineas}

	\item Escolha de provedores de serviços, dispositivos e ferramentas para o
desenvolvimento;

	\item Construir, avaliar, testar e manter os sensores;

	\item Construir o dispositivo agregador e sua API;

	\item Estimar o custo total do projeto piloto;

	\item Estimar o custo de replicação;

	\item Identificar os desafios para o desenvolvimento.

\end{alineas}

Desta forma, a liberdade necessária foi garantida para o projeto ser
executado com sucesso, mesmo no ambiente de incerteza no qual o mercado local de
IoT encontra-se, cumprindo as premissas de funcionamento, manutenção e
segurança que são grande importância para os interessados na área.
