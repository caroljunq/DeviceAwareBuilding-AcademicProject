\chapter{Resultados e Discussão}
\label{chap:resultados}

Neste capítulo, são abordados, analisados e discutidos os resultados encontrados
durante a exploração do tema e das plataformas e durante a implementação das
aplicações, além de verificar a precisão atingida com a aplicação implementada. Todos os testes foram realizados no prédio do laboratório
LTIA da Unesp de Bauru, onde os sensores permaneceram monitorando dispositivos.



\section{Método de teste}
\label{sec:metodo-teste}

Como discutido no \autoref{chap:Construcao}, a arquitetura geral da aplicação (\autoref{fig-arq-app})
mostra que a precisão vista na aplicação \emph{Web} depende dos resultados
encontrados pela aplicação sensor que, por sua vez, depende do par de
capacidades combinadas do \emph{hardware} adaptador Wi-Fi e do \emph{software}
\emph{TShark}. Portanto, a metodologia de testes empregada neste capítulo é
analisar diretamente as capacidades deste último par. Esta decisão também se
deve pela facilidade de armazenar e analisar arquivos CSV gerados pelo
\emph{TShark}.

As capturas foram executadas com o comando descrito no \autoref{code-tshark-pipe-assinc}
que é o mesmo utilizado na aplicação sensor.

\begin{lstlisting}[language=bash,caption={TShark e redirecionamento da saída para arquivo assíncrono},label=code-tshark-pipe-assinc]
pi@sensor-01:~ $ tshark -I -i wlan0 -T fields -E header=y -E quote=d \
-e wlan.sa -e wlan.sa_resolved -e wlan.ta -e wlan.ta_resolved \
-e radiotap.dbm_antsignal -e wlan_mgt.ssid \
>> 2017-01-17--02-48--rpi-02.csv &
pi@sensor-01:~ $ exit
\end{lstlisting}

Neste modo de uso, os resultados são direcionadas para a saída padrão
(stdout)  do terminal e podem ser capturados por outro programa no formato
de valores separados por vírgula (CSV). Os campos escolhidos para captura
são \emph{wlan.sa}, \emph{wlan.sa\_resolved}, \emph{wlan.ta},
\emph{wlan.ta\_resolved}, \emph{radiotap.dbm\_antsignal} e \emph{wlan\_mgt.ssid}.

Na linha 4 do \autoref{code-tshark-pipe-assinc}, o \emph{\&} representa o início
de um processo independente (assíncrono) e, a linha 5, a finalização do terminal.
Esta operação foi somente executada durante a captura longa.

A análise de dados foi feita com a função \emph{summary} da ferramenta
\emph{Ron’s editor}\footnote{\url{https://www.ronsplace.eu/Products/RonsEditor}}
e a filtragem com a função \emph{Filter} da ferramenta
\emph{RecCsvEditor}\footnote{\url{http://recsveditor.sourceforge.net/}}. Para a
construção dos gráficos, foi utilizada a ferramenta
\emph{WPS Spreadsheets}\footnote{\url{https://www.wps.com/office-free}}.
