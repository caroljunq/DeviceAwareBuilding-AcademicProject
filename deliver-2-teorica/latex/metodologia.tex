
\chapter{MÉTODO DE PESQUISA}
\label{chap:MÉTODO DE PESQUISA}

Abordagens para medir distâncias através de redes sem fio Wi-Fi
\cite{bahillo2009ieee} e Bluetooth já existem e, propor novas maneiras não é o
foco deste trabalho. Utilizando essas técnicas, propomos estabelecer uma rede de
nós sensores colaborativos fixos no ambiente onde deseja-se obter a localização
dos dispositivos. As informações de distância serão compartilhadas entre os nós
para maior precisão da informação.

Para a implementação, pretende-se utilizar os softwares de maior destaque
recentemente nos ramos de comunicação de baixa energia (\textit{MQTT}), serviços
\textit{Web} para armazenamento (\textit{MongoDB}) e publicação
(\textit{NodeJS}), além de softwares para medição da distância sem interferir na
comuncação (\textit{Sniffing}) e das plataformas de hardware disponíveis e
recomendadas para IoT com capacidade Wi-Fi e Bluetooth (\textit{Raspberry Pi
3}).

Mesmo com a grande quantidade de dispositivos já conectados são poucos os
documentos descrevendo boas práticas para concepção, construção e manutenção de
aplicações IoT, especialmente sobre os cuidados tomados quanto a segurança e
análise de custos para a implementação e manutenção.

Além disso, a falta de referências neste sentido é agravada quando considera-se
a implementação no interior do estado de São Paulo. Nesta região, poucas são as
organizações atualizadas neste tema, levando a uma falta enorme de conteúdo
escrito na linguagem local além de serviços e produtos disponíveis para
construção de uma plataforma completa e competitiva nesta região.

Devido a falta de conteúdo e instrução, utilizaremos prototipagem ágil para este
projeto, uma vez que esta metodologia de desenvolvimento é recomendada para
projetos cujas especificações e definições não são claras, demandando muitas
modificações das mesmas durante a execução do mesmo. Esse método entra em
contraste com metodologias clássicas, como a cascata que apesar de previsíveis,
não reagem bem a ambientes de extrema incerteza.

Mais especificamente, utilizaremos uma variante da metodologia \textit{Scrum}
\cite{James2016} que será adaptada para o projeto. Nela, serão executadas
iterações de uma semana em que a cada iteração, uma nova versão melhorada do
produto completo (hardware, software, documentação e resultados) será entregue.

Dentro de cada iteração, as camadas da aplicação IoT serão escolhidas,
implementadas, justificadas e avaliadas, sendo todo processo documentado. Como
resultado de cada uma delas, será gerado um relatório das mudanças a partir da
iteração anterior.

Com mais detalhes, cada iteração cumprirá uma parte de cada objetivo no trabalho
completo levando o projeto integralmente para um estágio de completude maior a
cada iteração. Serão foco de cada iteração os objetivos abaixo, gerando um
relatório utilizado para tomar e justificar decisões durante a execução do
projeto bem como servir de posterior documentação. Os objetivos de cada iteração
são:

\begin{alineas}

	\item Escolha de provedores de serviços, dispositivos e ferramentas para o
desenvolvimento;

	\item construir, avaliar, testar e manter dos sensores;

	\item construir um dispositivo agregador e sua API;

	\item Estimar o custo total do projeto piloto;

	\item Estimar o custo de replicação;

	\item Identificar os desafios para o desenvolvimento.

\end{alineas}

Desta forma, esperamos garantir a liberdade necessária para o projeto ser
executado com sucesso, mesmo no ambiente de incerteza no qual o mercado local de
IoT encontra-se, cumprindo as premissas de de funcionamento, manutenção e
segurança que são grande importância para os interessados na área.

