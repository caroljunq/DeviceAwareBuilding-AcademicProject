\section{Sensor}
\label{sec:app-sensor}

A aplicação sensor tem como requisitos funcionais capturar, avaliar e classificar pacotes de
Wi-Fi, inferir estatísticas de dispositivos e fornecer estas informações para
os interessados através do \emph{gateway}.

Para fazer a captura dos pacotes na aplicação final, diferente do que foi
demonstrado na \autoref{subsec:testes-rpi}, em especial o airodump-ng e
sua interface demonstrada na \autoref{fig-airodump}, foi utilizado o programa
TShark cujo modo de operação serve melhor para a construção dos
\emph{streams} que serão abordados em breve.

\begin{citacao}

	TShark é uma versão orientada ao terminal do Wireshark projetada para capturar
	e exibir pacotes quando uma interface de usuário interativa não é necessária ou
	disponível. Ele suporta as mesmas opções como wireshark. \

	\citeonline{tshark} Tradução Nossa.
\end{citacao}

Como foi estabelecido no capítulo anterior, TShark utiliza a saída padrão do
terminal (\emph{stdout}) como sua saída principal, esta característica foi
explorada com a aplicação Node.js. Mais especificamente, com o módulo
``child\_process'', que provê uma API que permite a criação e controle de
processos filhos do processo Node.js.

\begin{citacao}

	Node.js é uma estrutura em tempo de execução construida sobre o motor de
	execução JavaScript V8 do Chrome. Node.js utiliza um modelo orientado a
	evento, de entrada e saída não bloqueante que o faz leve e eficiente.
	O ecosistema de pacotes do Node.js, npm, é o maior ecossistema de bibliotecas
	de código livre no mundo. \

	\citeonline{nodejs} Tradução Nossa.
\end{citacao}

Como também foi estabelecido anteriormente, o TShark é executado com o
comando e argumentos como mostrado a seguir, a diferença em relação aos testes e
na escolha da plataforma é a forma de execução, na maneira mostrada, o processo é
criado utilizando o módulo ``child\_process'' \cite{child_process} e os
argumentos são passados como um vetor (\emph{Array}).

\begin{lstlisting}[language=javascript,caption={TShark e opções executado pelo Node.js},label=code-node-tshark]
const spawn = require('child_process').spawn;
const tsharkProcessoFilho = spawn(
	'tshark', [
		'-I',
		'-i', childIface,
		'-T', 'fields',
		'-E', 'separator=,',
		'-E', 'quote=d',
		'-e', 'wlan.sa',
		'-e', 'wlan.sa_resolved',
		'-e', 'wlan.ta',
		'-e', 'wlan.ta_resolved',
		'-e', 'radiotap.dbm_antsignal',
		'-e', 'wlan_mgt.ssid',
		'-Y', 'wlan.sa'
	]);
\end{lstlisting}

Para utilizar o resultado gerado pelo TShark, utilizou-se outro método do
módulo ``child\_process''  juntamente com a estrutura de \emph{Stream}
\cite{stream} que provê o método \texttt{pipe(destination[, options])} que permite,
de maneira análoga ao operador `` | '' no terminal também chamado de \emph{pipe},
redirecionar a saída de um processo ou \emph{stream} de leitura para outro
processo ou \emph{stream} de escrita.

Com isso, ainda falta um \emph{stream} de escrita que receba a saída do
TShark que foi definida no formato CSV. Para isso, uma biblioteca extra,
``fast-csv'', deve ser instalada. Com ela, pode-se criar o \emph{stream}
necessário e configurá-lo para interpretar os resultados \cite{fast-csv}.

\begin{lstlisting}[language=javascript,caption={Uso do fast-csv},label=code-node-csv]
const csv = require("fast-csv");
let csvStream = csv()
	.on("data", function(data){
		let packet = new Packet(
			data[0], // sender address
			data[1], // sender address resolved
			data[2], // transmitter address
			data[3], // transmitter address resolved
			data[4], // potencia de sinal (rss)
			data[5] // nome da rede no pacote Beacon
		);
		processarPacote(packet);
	})
	.on("end", function(){
		console.log("done with tshark");
	});
tsharkProcessoFilho.stdout.setEncoding('utf8');
tsharkProcessoFilho.stdout.pipe(csvStream);
\end{lstlisting}

Na linha 18, pode-se observar a operação de redirecionamento (\emph{pipe})
de \emph{stream} da saída padrão do processo filho (\emph{stdout}) para o
\emph{stream} de escrita descrito e configurado com o ``fast-csv''. A parte
faltante é o processamento dos pacotes e envio das inferências do sensor para o
\emph{gateway}.

Para as inferências e estatísticas, foi criado um objeto para armazenamento indexado
pelo endereço MAC e uma classe onde as informações
sobre cada dispositivo são agregadas. Desta forma, cada novo pacote pode ser acrescentado ao
histórico de cada dispositivo.


\begin{lstlisting}[language=javascript,caption={Adição do pacote ao histórico do dispositivo},label=code-device-packet]
let devices = {};	// lista indexada de dispositivos
class Device {
	[...]
	appendPacket(packet){
		let curTime = ( new Date() ).toISOString();
		this.rssHistory.push(packet.radiotap.dbm_antsignal);
		this.ssidHistory[packet.wlan_mgt.ssid] = curTime;
		this.taHistory[packet.wlan.ta] = {
			ta			: packet.wlan.ta,
			ta_resolved	: packet.wlan.ta_resolved,
		};
	}
}
function processarPacote(packet) {
	let sa = packet.wlan.sa;
	if (! devices[sa]){
		devices[sa] = new Device(sa, packet.wlan.sa_resolved);
	}
	devices[sa].appendPacket(packet);
}
\end{lstlisting}

A partir desta lista, é posível calcular a média e o desvio padrão de potência de
sinal para cada dispositivo descoberto.

\begin{lstlisting}[language=javascript,caption={Extração das estatísticas do dispositivo},label=code-device-stats]
class Device {
	[...]
	get rssStatistics(){
		let sum = 0, avg = 0, variance = 0, stdDeviation = 0;
		if (this.rssHistory.length > 0){
			for (let rss of this.rssHistory){
				sum += rss;
			}
			avg = sum / this.rssHistory.length;
			sum = 0;
			for (let rss of this.rssHistory){
				sum += Math.pow(( rss - avg ), 2);
			}
			variance = sum / (this.rssHistory.length - 1);
			stdDeviation = Math.sqrt(variance);
		}
		return {
			size			: this.rssHistory.length,
			avg				: avg,
			stdDeviation	: stdDeviation,
		};
	}
}
\end{lstlisting}


Por fim, é necessário comunicar aos interessados nessas inferências, o que é
feito através do módulo MQTT.js \cite{mqttjs}.

\begin{lstlisting}[language=javascript,caption={Cliente MQTT.js},label=code-mqttjs-client]
const mqtt = require('mqtt');
const mqttBrok = `mqtt://${config.mqttHost}:${config.mqttPort}`;
let clientMqtt	= mqtt.connect(mqttBrok, {
	username	: config.mqttUser,
	password	: config.mqttPwd,
})
clientMqtt.on('connect', function () {
	clientMqtt.subscribe('devices');
	startTshark();
});
clientMqtt.on('message', function (topic, message) {
	if (topic.toString() == 'devices') {
		switch (message.toString()) {
		case 'list':
			clientMqtt.publish(
				'devices/report',
				JSON.stringify( tshark.getDevices() )
			);
			break;
		case 'report':
			clientMqtt.publish(
				'devices/report',
				JSON.stringify( tshark.getReport() )
			);
			break;
		default:
			let mac = message.toString();
			clientMqtt.publish(
				'devices/report',
				JSON.stringify( tshark.getDeviceReport(mac) )
			);
			break;
		}
	}
});
\end{lstlisting}

No caso desta listagem de código, quando a aplicação é iniciada ela conecta-se ao \emph{MQTT Broker}.
Quando a conexão é estabelecida, ela solicita ao \emph{MQTT Broker} a inscrição
para receber as publicações no tópico ``devices'' e o processo filho TShark
é iniciado. O processo filho é executado, os resultados são capturados e
guardados.

Quando uma mensagem no tópico ``devices'' é recebida, três reações podem acontecer:
a lista de dispositivos deve ser fornecida, o relatório do sensor deve ser
fornecido ou um relatório sobre um dispositivo específico deve ser fornecido.
Para todos os casos, uma resposta adequada é imediatamente processada e enviada
para o tópico ``devices/report''.

Em conclusão, a aplicação sensor instancia o processo de aquisição de dados
TShark assincronamente e captura, classifica e armazena todos os pacotes
que ficam acessíveis através de requisições ao tópico MQTT ``devices''.
Isso cobre os requisitos funcionais desta aplicação.

Os requisitos não-funcionais desta aplicação estão ligados com o ambiente de
implantação que é um RPI3 remoto, sem outro dispositivo de entrada e saída
exceto a comunicação MQTT, portanto, sem nenhum aspecto permitindo a
monitoração por um humano que garanta que o aplicativo continue funcionando.

Este ambiente se assemelha muito a aplicações em núvem, onde não há acesso
físico ao computador onde a aplicação é executada. Neste ambiente, procura-se
garantir que a aplicação seja executada constantemente e atualizada
automaticamente sempre que uma nova versão é construída e testada. Estas
garantias podem ser alcançadas com ferramentas de integração contínua
(\emph{Continuous Integration} - CI). No caso do Github, a escolha de hospedagem
de código deste projeto, diversas opções deste tipo de ferramenta podem ser
encontradas\footnote{\emph{Integrations with Deployment}:\url{https://github.com/integrations/feature/deployment}}.

No entanto, para esta aplicação, construiu-se uma solução extremamente
simplificada destas ferramentas utilizando a arquitetura existente do programa
\emph{git} para garantir que novas versões fossem instaladas assim que possível.
A implementação consiste da adição, nas primeiras instruções do aplicativo,
uma rotina que executa sincronamente um programa externo através do terminal
(diferente da execução do TShark que é assíncrona). Este programa é
o \texttt{git} com a opção \texttt{pull} que, quando executado em um diretório que
é repositório, solicita ao servidor configurado, como \texttt{origin}, a atualização
do código do \emph{branch} atual. Se a mensagem encontrada na saída padrão
(\emph{stdout}) resultante deste comando for diferente de
\texttt{``Already up-to-date.''}, o programa finaliza imediatamente, pois uma nova
versão foi baixada e uma reinicialização da aplicação é necessária.
Em resumo, uma vez instalada a aplicação utilizando a ferramenta \texttt{git}, ela
será atualizada com o mesmo processo com que foi instalada.

O outro requisito é que a aplicação seja executada constantemente. Para isso,
pode-se utilizar uma aplicação também escrita em Node.js e disponível no
gerenciador de pacotes ``npm'': ``forever-service''. Ela registra um novo serviço no sistema operacional
linux para que a infra-estrutura de gerenciamento de serviços existente seja
aproveitada (por exemplo: início automático após o início do sistema operacional). Além disso, a sua dependência
``forever'' garante que a aplicação seja reinicializada sempre que finaliza por falha ou outra causa,
executando a aplicação constantemente \cite{forever-service}.
