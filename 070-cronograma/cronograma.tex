
\chapter{CRONOGRAMA}
\label{chap:CRONOGRAMA}

Devido a natureza ágil e iterativa da metodologia, o cronograma será dividido em
apenas três partes: Levantamento Bibliográfico Inicial, Desenvolvimento
Iterativo (Escolha de provedores e fornecedores; Construção, avaliação, teste e
manutenção dos sensores e agregadores; Estimativas de custos totais e de
replicação e Documentação de desenvolvimento) e Revisão Final. Estas partes
serão distribuídas durante o ano letivo conforme a Tabela 1 considerando as
alterações do calendário letivo da Faculdade de Ciências da UNESP de Bauru que
esteve em estado de greve de 1º de Junho à 18 de Agosto de 2016.

\begin{table}[htb]
\IBGEtab{%
\ABNTEXchapterfont {
  \caption{Cronograma de Atividades Propostas}%
  \label{table:cronograma}
}
}{%
  \begin{tabular}{cccccccccccc}
  \toprule
	Atividade															&	Fev	&	Mar	&	Abr	&	Mai	&	Ago	&	Set	&	Out	&	Nov	&	Dez	&	Jan	&	Total	\\
  \midrule \midrule
	Levantamento Bibliográfico \\ Inicial								&	X	&	X	&	 	&	 	&	 	&	 	&	 	&	 	&	 	&		&	2	\\
	\midrule
	Escolha de \\ provedores e fornecedores								&	 	&	X	&	X	&	X	&	X	&	X	&	X	&		&		&		&	6	\\
	\midrule
	Construção, avaliação e\\manutenção  dos sensores \\ e agregadores	&	 	&		&	X	&	X	&	X	&	X	&	X	&	X	&		&		&	6	\\
	\midrule
	Estimativas de custos												&	 	&		&		&		&	X	&	X	&	X	&	X	&	X	&		&	5	\\
	\midrule
	Documentação de \\ desenvolvimento									&	 	&		&	X	&	X	&	X	&	X	&	X	&	X	&	X	&		&	7	\\
	\midrule
	Revisão Final														&	 	&	 	&	 	&	 	&	 	&	 	&	 	&	X 	&	X	&	X	&	3	\\
	\midrule \midrule
	Semanas disponíveis													&	4 	&	5 	&	4 	&	4 	&	2 	&	 4	&	4 	&	4 	&	3	&	3	&	37	\\
	\midrule
	Total de atividades													&	4 	&	10 	&	12 	&	12 	&	8 	&	 16	&	16 	&	16 	&	9	&	3	&	106	\\
	\midrule
	\bottomrule
\end{tabular}%
}{%
  \fonte{Produzido pelo autor.}%
  }
\end{table}

Cada atividade realizada representa uma melhoria ou avanço no projeto total
porém o número total é apenas uma estimativa devido a natureza da metodologia.
Nesta segunda versão do cronograma foram estimadas 106 atividades distribuidas
em 37 semanas, de acordo com as alterações dos calendários letivos anteriormente
mencionados. As diferenças entre as duas vesões são nos meses de junho a janeiro
onde houve redução de 40 para 37 semanas disponíveis e consequente redução da
estimativa de atividades de 111 para 106.
