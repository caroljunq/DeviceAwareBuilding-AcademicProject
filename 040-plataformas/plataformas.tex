\chapter{Plataformas}
\label{chap:Plataformas}

Para a localização com os residuos de comunicação \emph{WiFi} são necessários
sensores que possam capturar estes residuos e processar qualquer informação
capturada por esse sensor. Esta plataforma de sensor pode ser construida com
qualquer plataforma computacional capaz de ser programada com comunicação
\emph{WiFi} porém o \emph{hardware} de \emph{WiFi} e seu \emph{software}
controlador deve permitir o Modo Promíscuo.

Este Modo Promíscuo (\emph{promiscuous mode}) é definindo pela capacidade de uma
Placa Adaptadora de Rede \emph{WiFi} (\emph{Network Interface Card} -
\emph{NIC}) receber e interpretar todos os pacotes que trafegam em uma rede ou
em todas as redes que estão em seu alcance, independentemente do destinatário do
pacote. Em seu fucionamento normal uma \emph{NIC} descarta todos os pacotes que
não são destinados para ela o mais cedo possível evitando reprocessamento de
dados indesejáveis, por este motivo não são todas as \emph{NICs} que permitem o
Modo Promíscuo Essa funcionalidade elimina a necessidade de \emph{hardware} ou
\emph{software} em cada um dos dispositivos rasterados.

Neste sentido elegeram duas plataformas de notável importância no mercado atual
e notável facilidade de acesso para qualquer interessado na área. As plataformas
testadas são o microcomputador \emph{Raspberry Pi} e o microcontrolador
\emph{ESP8266}. Ambos  foram escolhidos pelo domínio do segmento de Prototipação
e Faça Você Mesmo  (\emph{Do It Yourself} - \emph{DYI}) dentro do campo de IoT.
Outro líder de segmento, o \emph{Arduino}  foi prontamente descartado por não
conter nativamente a habilidade de conectar-se à \emph{internet} sendo
constantemte combinado com um dos escolhidos para ganhar esta habilidade
demonstrando claramente menor afinidade a este projeto em comparação aos seus
igualmente famosos concorrentes.
