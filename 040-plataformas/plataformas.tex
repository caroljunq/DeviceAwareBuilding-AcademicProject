\chapter{Plataformas}
\label{chap:Plataformas}

Para a localização com os resíduos de comunicação \emph{Wi-Fi} são necessários
sensores que possam capturar estes resíduos e processar qualquer informação
capturada pelo sensor deste trabalho. Esta plataforma de sensor pode ser construída com
qualquer plataforma computacional capaz de ser programada com comunicação
\emph{Wi-Fi}, porém o \emph{hardware} de \emph{Wi-Fi} e seu \emph{software}
controlador deve permitir o Modo Promíscuo.

Este Modo Promíscuo (\emph{promiscuous mode}) é definindo pela capacidade de uma
Placa Adaptadora de Rede \emph{Wi-Fi} (\emph{Network Interface Card} -
\emph{NIC}) receber e interpretar todos os pacotes que trafegam em uma rede ou
em todas as redes que estão em seu alcance, independentemente do destinatário do
pacote. Em seu funcionamento normal, uma \emph{NIC} descarta todos os pacotes que
não são destinados para ela o mais cedo possível, evitando reprocessamento de
dados indesejáveis, por este motivo não são todas as \emph{NICs} que permitem o
Modo Promíscuo. Essa funcionalidade elimina a necessidade de \emph{hardware} ou
\emph{software} em cada um dos dispositivos rastreados.

Neste sentido, elegeu-se duas plataformas de notável importância no mercado atual
e notável facilidade de acesso para qualquer interessado na área. As plataformas
testadas foram o microcomputador \emph{Raspberry Pi} e o microcontrolador
\emph{ESP8266}. Ambos  foram escolhidos pelo domínio do segmento de Prototipação
e Faça Você Mesmo  (\emph{Do It Yourself} - \emph{DIY}) dentro do campo de IoT.
Outro líder de segmento, o \emph{Arduino}  foi prontamente descartado por não
conter nativamente a habilidade de conectar-se à \emph{Internet} sendo
constantemte combinado com um dos escolhidos para ganhar esta habilidade,
demonstrando claramente menor afinidade a este projeto em comparação aos seus
igualmente famosos concorrentes.

Após escolhidas as plataformas de intersse alguns exemplares de cada uma delas
foi adquirido para implementar a aplicação proposta. Neste sentido, serão
apresentadas cada uma dessas plataformas quanto as suas especificações técnicas
e aos produtos utilizados em conjunto para que elas pudessem funcionar e serem
programadas e os motivos pela adoção ou não delas.
