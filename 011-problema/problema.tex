\section{Problema}
\label{sec:Problema}

Tamanha quantidade de dispositivos conectados pouco acrescenta na vida diária se
humanos ou coisas não puderem simplesmente se encontrar. Tanto em ambiente real
quanto virtual é essencial o contato e conhecimento entre as partes envolvidas
para que uma interação complexa seja executada. Portanto, para que uma aplicação
IoT funcione corretamente, o conhecimento do contexto em que todos os
interessados, sejam coisas ou pessoas, estão inseridos é indispensável. Para a
maioria das aplicações, a informação contextual de maior relevância é a
localização física.

Em situações em que a localização contextual é essencial para o bom funcionamento
de uma aplicação IoT, destaca-se a necessidade da coleta desta informação através
de sensores ativos sempre que a aplicação requisite a ciência deste contexto
em suas tomadas de decisão. E, também, para que outros (sistemas, pessoas e coisas)
saibam a localização de qualquer dispositivo ao qual têm interesse de interagir,
distribuindo efetivamente essa informação coletada sobre o contexto com todos os
que se encontram envolvidos no mesmo contexto.

Um exemplo desta necessidade de localização de dispositivos dentro de um prédio
seria um profissional saber onde está o dispositivo em seu local de trabalho,
seja ele um vendedor e seu \emph{tablet} para demostrar um produto fora de
estoque em uma loja ou um médico e seu equipamento portátil.

\subsection{Sobre Sistemas de Posicionamento}
\label{subsec:Sobre Sistemas de Posicionamento}

Sistemas de posicionamento (PS - \emph{Positioning System}) são geralmente
constituídos de um Ponto Origem Global escolhido (\emph{O}) e um conjunto não
vazio de Pontos de Referência (RP - \emph{Reference Point}) cuja localização
global em relação ao \emph{O} é conhecida com uma certa precisão quando o sistema
é construído - precisão de construção. Então, para o usuário, um sistema de posicionamento
oferece como resultado uma precisão de visualização menor que a sua precisão de construção.
Um PS tem interesse em determinar a posição de um ponto móvel (MU - \emph{Mobile User}).
Essa localização é feita encontrando um conjunto de distâncias associadas a cada um dos RPs em um
sub-conjunto com dimensão variável de acordo com o método utilizado. Feito isso, é
possível utilizar modelos matemáticos para, a partir das distâncias, encontrar
uma posição do MU em relação aos RPs e uma nova transformação é aplicada para
encontrar a posição relativa ao \emph{O}.

Uma das maneiras de classificar PSs é entre as classes de Auto Posicionamento e
Posicionamento Remoto. Os de Auto Posicionamento contém no MU todo aparato
necessário para medir a distância dos RPs e calcular a posição em relação a
\emph{O}. Já os classificados como de Posicionamento Remoto tem o mínimo
necessário na MU e todo o trabalho de cálculo de distância e posição global é
feito nos RPs ou em uma unidade coordenadora destes.

Para PSs eletrônicos baseados em radio-frequência (RF - \emph{Radio
Frequency}), geralmente, utilizam-se dois componentes básicos, Transmissores e
Receptores, os quais assume-se que ao menos um destes está no RP e ao menos um
outro no MU. Para calcular a distância entre MU e RP, utiliza-se as propriedades
da comunicação por RF como tempo de chegada (TOA - \emph{Time Of Arrival}),
diferencial de tempo de chegada (TDOA - \emph{Time Difference Of Arrival}) e
ângulo de chegada de sinal (AOA - \emph{Angle Of Arrival}).

Para maior precisão, é comum a utilização de múltiplas RPs geralmente com o
número mínimo igual ao número de dimensões espaciais que deseja-se calcular.
Nota que para sistemas distribuídos a sincronização de relógios é um problema
intrínseco, então é fundamental que o tempo seja incluído como dimensão.

Os sistemas classificados como ``Sistema de Navegação Global por Satélite''
(GNSS - \emph{Global Navigation Satellite System}), como o tradicional
estadunidense Sistema de Posicionamento Global (GPS - \emph{Global Positioning
System}), utilizam a técnica em que o dispositivo móvel contém o receptor e os
transmissores são fixos em satélites na órbita terrestre \cite{Djuknic2001}.
Devido a posição e número de satélites, o GPS e seus correlatos estão sempre
presentes do ponto de vista de um observador da superfície terrestre, sendo para
este tipo de usuário um sistema ubíquo.

Entretanto, a força do sinal GNSS não é suficiente para penetrar a maioria dos
prédios, uma vez que estes dependem de visão direta (LOS -
\emph{Line-Of-Sight}) entre os satélites e o receptor. A reflexão do sinal
muitas vezes permite a leitura em ambientes fechados, porém o cálculo da posição
não será confiável \cite{Chen2000}. Logo, apesar da ubiquidade dos
GNSSs em ambientes abertos, são necessárias soluções diferentes para obter um
Sistema de Posicionamento para Ambientes Fechados (IPS - \emph{Indoor
Positioning System}), sendo a ubiquidade deste essencial para conquistar o mesmo
nível de confiança trazido pelos GNSSs.

Para implementar este IPS, propõem-se o uso de tecnologias já implantadas em
dispositivos móveis e essenciais para o funcionamento dos mesmos, especialmente
as de camadas de comunicação, que são ubíquas no ambiente dos dispositivos
móveis, como Wi-Fi (padrão \emph{IEEE 802.11}) e \emph{Bluetooth}
(padrão \emph{Bluetooth SIG}), para que os objetos que deseja-se obter a localização
contextual não necessitem de modificações.

Outros protocolos de comunicação sem fio ubíquos existem (em especial, o
celulares em todas as gerações 2G, 3G, 4G), porém não oferecem a mesma
flexibilidade por trabalharem em uma faixa de radio-frequência licenciada e por
questões de propriedade da rede que serão abordadas na seção de Localização
Contextual desta mesma obra.

De forma semelhante, existem protocolos mais flexíveis (nas faixas não
licenciadas como \emph{NFC}, infra-vermelho, \emph{ZigBee} ou
\emph{SIGFOX}), porém estes não estão presentes na maioria dos aparelhos
utilizados, tanto global quanto localmente, removendo a característica da
forma de comunicação ubíquoa que é foco deste trabalho.

Devido às restrições anteriores, justifica-se o foco deste trabalho em tecnologias de
comunicação Wi-Fi e \emph{Bluetooth}. Ambas as tecnologias
tem interesse para esta obra, pois, a nível global, elas possuem mesma importância e presença no mercado atual, permitem
flexibilidade por possuírem protocolos conhecidos por todos em frequências
livres de licenciamento e dentro da área de cobertura que são de nosso interesse
e o usuário final já ser o proprietário da rede local criada. Porém, trabalhar com as duas
tecnologias simultaneamente é um problema complexo por si só, então, a escolha
de uma ou outra deve ser feita. Para o presente trabalho, escolheu-se a tecnologia Wi-Fi, visto que está sempre
ligado em todos os dispositivos, conectando-os à Internet, enquanto o \emph{Bluetooth} tende
a ser mantido desligado. Logo, por considerar como fator decisivo a observação do ambiente teste do protótipo
desenvolvido, o Wi-Fi apresenta-se como opção de maior interesse.
