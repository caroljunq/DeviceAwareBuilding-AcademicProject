\chapter{Fundamentação teórica}
\label{chap:Fundamentação teorica}


Para conceituar, fundamentar e dar suporte teórico ao presente trabalho
apresentam-se nesta seção os tópicos e definições dos segmentos: IoT,
localização contextual de dispositivos e localização baseada em redes sem fio.

\section{Internet das coisas (IoT)}
\label{sec:INTERNET DAS COISAS (IOT)}

Uma das primeiras aplicações e definições de IoT foi feita por Kevin Ashton em
1999 para a \textit{Procter \& Gamble} (P\&G) \cite{ASHTON2009} e
simultaneamente no laboratório Auto-ID Labs no Instituto de Tecnologia de
Massachusetts (MIT - \textit{Massachusetts Institute of Technology}) utilizando
identificação por radio-frequência (RFID - \textit{radio-frequency
identification}) \cite{ATZORI2010, Friedemann2011} e desde então cresceu
ultrapassando o escopo da tecnologia RFID porém sempre com as premissas de ``uma
infraestrutura global para a Sociedade da Informação, habilitando serviços
avançados através da interconexão de coisas (físicas e virtuais) baseadas em
tecnologias, existentes e evolutivas, de informação e comunicação'' descrita por
\apudonline[p.~1, grifo e tradução
nossa]{Wortmann2015}{InternationalTelecommunicationUnion2012}.

Hoje em dia, quase qualquer tecnologia de comunicação acessível a computadores
pode ser utilizada como meio de comunicação entre dispositivos IoT, tornando
RFID mais uma, porém de grande importância, tecnologia info-comunicacional a
disposição das coisas para sua conexão. Esta gama de tecnologias possibilita uma
variedade equivalente de coisas conectadas. Se a coisa pode usar de uma
tecnologia de conexão, considerando suas restrições de volume, custo e
utilidade, muito provavelmente vai fazê-lo gerando ao menos uma identidade
virtual representando seu objeto físico e seus atributos. Esta identidade
virtual e atributos virtuais serão expostos para todos indivíduos, humanos ou
coisas, que lhe forem convenientes de qualquer lugar do universo virtual,
fazendo efetivamente parte da internet.

\section{Localização contextual de dispositivos}
\label{sec:Localização contextual de dispositivos}

Em ciência da computação, os termos \textit{"Contexto"} e \textit{"Consciência
de Contexto"} expressam uma ideia recente estudada nos campos de inteligência
artificial e ciência cognitiva desde 1991. O tema "Contexto" ainda é considerado
atual e promissor a ponto de mudar o cenário de negócios nos próximos 10 anos
porém sem definição simples. Tamanha é a falta de uma definição geral que
realmente funcione para casos reais que existe uma proposta de definir o termo
utilizando uma nova metodologia de pesquisa holística através de mineração e
agrupamento de texto advindo de publicações cientificas \cite{Pascalau2013}.

Mesmo sem uma definição permanente em vista, utilizaremos o que é considerado
estado da arte para o termo \textit{"Contexto"} que foi introduzido por
\citeonline{Dey1999} e reforçado por \citeonline{Dey2000}:

\begin{citacao}

	``Contexto é qualquer informação que pode ser utilizada para caracterizar a
	situação de uma entidade. Uma entidade é uma pessoa, lugar ou objeto que é
	considerado relevante para a interação entre um usuário e uma aplicação,
	incluindo o próprio usuário e a aplicação.'' \

	\citeonline[p.~3]{Dey1999} Tradução Nossa.
\end{citacao}

\subsection{Localização contextual}
\label{subsec:Localização contextual}

Das informações contextuais que uma aplicação de cliente móvel pode obter, a
localização é uma das mais importantes. Ajudar pessoas a navegar por mapas,
encontrar objetos e pessoas com os quais tem interesse de interagir é sem dúvida
uma boa meta a ser alcançada com a coleta da localização do cliente
\cite{Bellavista2008}.

Na categoria de Serviços Baseados em Localização (LBS - \textit{Location-Based
Services}) existem duas gerações. A primeira orientada a conteúdo que falhou,
pois a informação de localização era armazenada pela rede (que geralmetne era
administrada por uma empresa de telecomunicações), podendo até ser vendida pelo
provedor a terceiros, causando a sensação de \textit{Spam} (conteúdo não
solicitado) no usuário final ao receber conteúdo desta provedora. Já na segunda
geração, a posse da informação foi movida para o cliente móvel, deixando a cargo
do usuário escolher se ela seria compartilhada e com quem. Esta mudança trouxe
maior engajamento do usuário, resultando numa maior aceitação dessa geração
\cite{Bellavista2008}.


Ao contrário das técnicas atuais, neste trabalho os humanos ou tomadores de
decisão não estarão em posse do cliente móvel, e sim em posse do prédio.
Portanto, a mesma informação, sem degradação em sua importância, passará a ser
coletada e armazenada pelo provedor da rede como nos LBSs de primeira geração.
Esta decisão garante o foco no usuário uma vez que este mudou, antes ele detinha
um cliente móvel, agora ele detem múltiplos. Isso torna a detenção do todo
(coisas dentro do prédio) mais precioso do que o das partes (os clientes móveis)
além da mudança da propriedade da rede para o usuário final, na comparação
celular \textit{versus} \textit{Wi-Fi}.

Uma vez encontrada a localização de um dispositivo, metadados sobre o prédio são
mesclados formando um conjunto rico contextualmente do ponto de vista da
aplicação IoT Prédio como fornecedora principal dos dados para a internet e
portanto seus usuários detentores. Essa riqueza é garantida com metadados sobre
o dispositivo (identificação, nome, histórico, carecteristicas) e sobre o prédio
(ex.: mapa, estrutura de salas, horário de funcionamento, consumo energético,
humanos responsáveis e lista de equipamentos) que trazem possibilidades de
extração de informação importantes para os detentores deste prédio e seu
conteúdo. Esta capacidade do prédio deve-se pelo papel de coordenador de
informações e controlador de meta-informações semelhante ao Coordenador em uma
aplicação na arquitetura Modelo-Apresentação-Adaptador-Controlador-Coordenador
(MPACC - \textit{Model-PresentationAdapter-Controller-Coordinator}) proposto por
\citeonline{Roman2001}.


\subsection{Contexto de um dispositivo em um prédio}
\label{subsec:Contexto de um dispositivo em um prédio}

Para os metadados agregados à informação de posição pelo prédio definimos que o
modelo de divulgação terá de conter além da posição do dispositivo informação
sobre este (nome, histórico), informação da estrutura do prédio (mapa imagem,
mapa lógico, nome, localização global, endereço, etc), ligação entre a estrutura
do prédio e a localização do dispositivo (posição no mapa lógico) e informação
sobre o estado do prédio (horário de funcionamento, frequentadores, etc).


Este modelo visa prover fácil mineração e reutilização de informações por
terceiros após a implementação do projeto que é medida pela disponibilidade e
relacionamento das informações providas. Essa métrica também será utilizada para
avaliar o projeto final.

Este foco em reusabilidade vem da definição de Web Semantica (\textit{Semantic
Web}) e de uma de suas realizadoras, a Ligação de Dados (\textit{Linked Data}),
que sugerem o uso de um formato padrão além de ser acessível e gerenciável pelas
ferramentas de exploração. Desta forma a Web de Dados (\textit{Web of Data}) é
construída opondo uma simples coleção de dados \cite{Bizer2009}.

\section{Localização baseada em redes sem fio}
\label{sec:Localização baseada em redes sem fio}

Para o sistema de posicionamento nos baseamos em técnicas de
\textit{n-lateração?} de distâncias adquiridas com a medição de características
eletromagnéticas (ex.: potência de sinal) e dos protocolos (ex.: Tempo de
chegada) que já foram explorados anteriormente \cite{Abusubaih2007,
bahillo2009ieee, Feldmann2003}.

Portanto os sensores seguirão as especificações de \textit{WiFi IEEE 802.11}
\cite{Crow1997} e técnicas definidas para \textit{Bluetooth Low Energy (BLE)}
\cite{Hossain2007} devido a semelhança da área de cobertura (até 100 metros,
geralmente utilizado até 20 metros) e frequência (no caso de 2.4GHz).

Para construir estes sensores uma plataforma de hardware adequada é necessária,
para esta escolhemos o Raspberry Pi \cite{Vujovic2014, Vujovic2015} que já
foi provado funcional no caso de Localização através \textit{Wi-Fi} por
\citeonline{Ferreira2016} especialmente a sua versão 3 que adiciona a capacidade
de sensor \textit{Wi-Fi} e \textit{Bluetooth} em sua placa principal sem
necessidade de adaptadores externos destacando ainda mais sua escolha
\cite{RPI2016}.

Em adição, na construção dos sensores é testada a plataforma ESP-8266 bem como
outras alternativas que demonstrarem afinidade com o projeto não limitando as
escolhas durante o projeto a aquelas mencionadas na proposta original.
