\chapter[INTRODUÇÃO]{INTRODUÇÃO}
%\addcontentsline{toc}{chapter}{Introdução}

Nos recentes anos de 2014 à 2016, Internet das Coisas (IoT - \textit{Internet of
Things}) vem tomando o foco das atenções de empresas e entusiastas de Tecnologia
da Informação \cite{DzoneIoT:2015} a tal ponto que as empresas líderes do
segmento já incluem IoT como uma de suas áreas de atuação \cite{Ibm2016,
ARM-mbed, Microsoft2016, Intel2016, Oracle2016, Google2016, AmazonIoT2016}.

Todo este movimento no mercado é justificado pelo baixo custo dos pequenos
dispositivos computacionais \cite{RpiZeroLaunch, Esp8266.net} e grandes
serviços na nuvem \cite{Kaufmann2015, Amazon2016}. Este baixo custo
possibilita a computação ubíqua descrita por Weiser em 1991 e 1992
\cite{Weiser1999} que é entendida pelos autores como \textit{``computação
virtualmente onipresente''}. Também para os autores, esta virtual onipresença é
base e consequência para a IoT, sendo esta a realizadora da computação ubíqua.

Uma vez contextualizado o mercado e a oportunidade de implementação da
computação ubíqua, percebe-se a necessidade de dar aos elementos cotidianos
(coisas) a capacidade info-computacional, tornando-os sensores e atuadores
conectados, unicamente identificáveis e acessíveis através da rede mundial de
computadores \cite{Lemos2013, Kranenburg2012}.

É esperado que uma quantia total de 6,4 bilhões de dispositivos conectados
exista até o final de 2016 \cite{GARTNER2015} e entre 26 bilhões
\cite{GARTNER2014} e 50 bilhões até 2020 com até 250 novas coisas conectando-se
por segundo \cite{CiscoBlog2013}.
