% resumo em português
\setlength{\absparsep}{18pt} % ajusta o espaçamento dos parágrafos do resumo
\begin{resumo}

	IoT é o foco de empresas e entusiastas devido ao seu incrível crescimento com
	milhares de novos dispositivos todos os dias. Tudo isso construído sobre os
	baixos custos de processamento tanto em pequenos dispositivos quanto em
	grandes núvens e da capacidade comunicacional que é cada vez mais exigida e
	presente em coisas do dia-a-dia.
	Através da exploração de plataformas emergentes (como o ESP8266 e Raspberry
	Pi) e da construção de protótipos, este trabalho teve como objetivo
	construrir um sensor que permita que um prédio localize contextualmente
	qualquer dispositivo que se comunique utilizando \emph{Wi-Fi}. Para
	alcançar esse objetivo, utilizou-se diversas ferramentas tencnológicas,
	incluindo Raspberry Pi 3, TShark, Node.js e MQTT. Estas ferramentas
	possibilitaram testes onde confirmou-se que não é possível associar uma
	distância geográfica à potência de sinal recebida (RSS) no caso de
	comunicações \emph{Wi-Fi}, porém, com o mesmo sensor, é possível associar um
	dispositivo ao contexto de um sensor como uma sala dentro de um prédio.

	\textbf{Palavras-chave}: Internet das Coisas. Raspberry Pi. Localização Contextual. MQTT.
\end{resumo}
