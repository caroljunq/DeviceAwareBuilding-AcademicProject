% resumo em inglês
\begin{resumo}[Abstract]
\begin{otherlanguage*}{english}

	IoT is at the focus of companies and enthusiasts due to its incredible
	growth with thousands of new devices every day. All built on top of the low
	processing costs (in both small hardware and large clouds) and the
	communicational capacity that is increasingly required by businesses and
	consumers alike and present in everyday things.
	Through the exploration of emerging platforms (such as ESP8266 and Raspberry
	Pi) and the construction of prototypes, this work aimed to construct a
	sensor that allows a building to contextually locate any device that
	communicates using \emph{Wi-Fi}.
	To achieve this goal, several technological tools were used, including
	Raspberry Pi 3, TShark, Node.js and MQTT. These tools enabled tests where it
	was confirmed that it is not possible to associate a geographic distance to
	received signal strength (RSS) in the case of \emph{Wi-Fi} communications, but
	with the same sensor we conclude that it is possible to associate a device
	with the context of that sensor such as at a room inside a building.

	\textbf{Keywords}: Internet of Things. Raspberry Pi. Contextual location. MQTT.
\end{otherlanguage*}
\end{resumo}
